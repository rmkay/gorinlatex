\chapter{THE FIRE BOOK}

In this the Fire Book of the Ni To Ichi school of strategy I describe fighting as fire.\\

In the first place, people think narrowly about the benefit of strategy. By using only their fingertips, they only know the benefit of three of the five inches of the wrist. They let a contest be decided, as with the folding fan, merely by the span of their forearms. They specialise in the small matter of dexterity, learning such trifles as hand and leg movements with the bamboo practice sword.\\

In my strategy, the training for killing enemies is by way of many contests, fighting for survival, discovering the meaning of life and death, learning the Way of the sword, judging the strength of attacks and understanding the Way of the "edge and ridge" of the sword.\\

You cannot profit from small techniques particularly when full armour is worn. My Way of Strategy is the sure method to win when fighting for your life one man against five or ten. There is nothing wrong with the principle "one man can beat ten, so a thousand men can beat ten thousand". You must research this. Of course you cannot assemble a thousand or ten thousand men for everyday training. But you can become a master of strategy by training alone with a sword, so that you can understand the enemy's strategy, his strength and resources, and come to appreciate how to apply strategy to beat ten thousand enemies.\\

Any man who wants to master the essence of my strategy must research diligently, training morning and evening. Thus can he polish his skill, become free from self, and realize extraordinary ability. He will come to possess miraculous power.\\

This is the practical result of strategy.\\
\section{Depending on the Place}

Examine your environment.\\

Stand in the sun; that is, take up an attitude with the sun behind you. If the situation does not allow this, you must try to keep the sun on your right side. In buildings, you must stand with the entrance behind you or to your right. Make sure that your rear is unobstructed, and that there is free space on your left, your right side being occupied with your side attitude. At night, if the enemy can be seen, keep the fire behind you and the entrance to your right, and otherwise take up your attitude as above. You must look down on the enemy, and take up your attitude on slightly higher places. For example, the Kamiza in a house is thought of as a high place.\\

When the fight comes, always endeavour to chase the enemy around to your left side. Chase him towards awkward places, and try to keep him with his back to awkward places. When the enemy gets into an inconvenient position, do not let him look around, but conscientiously chase him around and pin him down. In houses, chase the enemy into the thresholds, lintels, doors, verandas, pillars, and so on, again not letting him see his situation.\\

Always chase the enemy into bad footholds, obstacles at the side, and so on, using the virtues of the place to establish predominant positions from which to fight. You must research and train diligently in this.\\
\section{The Three Methods to Forestall the Enemy}

The first is to forestall him by attacking. This is called Ken No Sen (to set him up).\\

Another method is to forestall him as he attacks. This is called Tai No Sen (to wait for the initiative).\\

The other method is when you and the enemy attack together. This is called Tai Tai No Sen (to accompany him and forestall him).\\

There are no methods of taking the lead other than these three. Because you can win quickly by taking the lead, it is one of the most important things in strategy. There are several things involved in taking the lead. You must make the best of the situation, see through the enemy's spirit so that you grasp his strategy and defeat him. It is impossible to write about this in detail.\\
\section{The First - Ken No Sen}

When you decide to attack, keep calm and dash in quickly, forestalling the enemy. Or you can advance seemingly strongly but with a reserved spirit, forestalling him with the reserve.\\

Alternatively, advance with as strong a spirit as possible, and when you reach the enemy move with your feet a little quicker than normal, unsettling him and overwhelming him sharply.\\

Or, with your spirit calm, attack with a feeling of constantly crushing the enemy, from first to last. The spirit is to win in the depths of the enemy.\\

These are all Ken No Sen.\\
\section{The Second - Tai No Sen}

When the enemy attacks, remain undisturbed but feign weakness. As the enemy reaches you, suddenly move away indicating that you intend to jump aside, then dash in attacking strongly as soon as you see the enemy relax. This is one way.\\

Or, as the enemy attacks, attack still more strongly, taking advantage of the resulting disorder in his timing to win.\\

This is the Tai No Sen principle.\\
\section{The Third - Tai Tai No Sen}

When the enemy makes a quick attack, you must attack strongly and calmly, aim for his weak point as he draws near, and strongly defeat him.\\

Or, if the enemy attacks calmly, you must observe his movements and, with your body rather floating, join in with his movements as he draws near. Move quickly and cut him strongly.\\

This is Tai Tai No Sen.\\

These things cannot be clearly explained in words. You must research what is written here. In these three ways of forestalling, you must judge the situation. This does not mean that you always attack first; but if the enemy attacks first you can lead him around. In strategy, you have effectively won when you forestall the enemy, so you must train well to attain this.\\
\section{To Hold Down a Pillow}

"To Hold Down a Pillow" means not allowing the enemy's head to rise.\\

In contests of strategy it is bad to be led about by the enemy. You must always be able to lead the enemy about. Obviously the enemy will also be thinking of doing this, but he cannot forestall you if you do not allow him to come out. In strategy, you must stop the enemy as he attempts to cut; you must push down his thrust, and throw off his hold when he tries to grapple. This is the meaning of "to hold down a pillow". When you have grasped this principle, whatever the enemy tries to bring about in the fight you will see in advance and suppress it. The spirit is too check his attack at the syllable "at...", when he jumps check his jump at the syllable "ju...", and check his cut at "cu...".\\

The important thing in strategy is to suppress the enemy's useful actions but allow his useless actions. However, doing this alone is defensive. First, you must act according to the Way, suppressing the enemy's techniques, foiling his plans and thence command him directly. When you can do this you will be a master of strategy. You must train well and research "holding down a pillow".\\
\section{Crossing at a Ford}

"Crossing at a ford" means, for example, crossing the sea at a strait, or crossing over a hundred miles of broad sea at a crossing place. I believe this "crossing at a ford" occurs often in man's lifetime. It means setting sail even though your friends stay in harbour, knowing the route, knowing the soundness of your ship and the favour of the day. When all the conditions are meet, and there is perhaps a favourable wind, or a tailwind, then set sail. If the wind changes within a few miles of your destination, you must row across the remaining distance without sail.\\

If you attain this spirit, it applies to everyday life. You must always think of crossing at a ford.\\

In strategy also it is important to "cross at a ford". Discern the enemy's capability and, knowing your own strong points, "cross the ford" at the advantageous place, as a good captain crosses a sea route. If you succeed in crossing at the best place, you may take your ease. To cross at a ford means to attack the enemy's weak point, and to put yourself in an advantageous position. This is how to win large-scale strategy. The spirit of crossing at a ford is necessary in both large- and small-scale strategy.\\

You must research this well.\\
\section{To Know the Times}

"To know the times" means to know the enemy's disposition in battle. Is it flourishing or waning? By observing the spirit of the enemy's men and getting the best position, you can work out the enemy's disposition and move your men accordingly. You can win through this principle of strategy, fighting from a position of advantage.\\

When in a duel, you must forestall the enemy and attack when you have first recognised his school of strategy, perceived his quality and his strong and weak points. Attack in an unsuspecting manner, knowing his metre and modulation and the appropriate timing.\\

Knowing the times means, if your ability is high, seeing right into things. If you are thoroughly conversant with strategy, you will recognise the enemy's intentions and thus have many opportunities to win. You must sufficiently study this.\\
\section{To Tread Down the Sword}

"To tread down the sword" is a principle often used in strategy. First, in large scale strategy, when the enemy first discharges bows and guns and then attacks it is difficult for us to attack if we are busy loading powder into our guns or notching our arrows. The spirit is to attack quickly while the enemy is still shooting with bows or guns. The spirit is to win by "treading down" as we receive the enemy's attack.\\

In single combat, we cannot get a decisive victory by cutting, with a "tee-dum tee-dum" feeling, in the wake of the enemy's attacking long sword. We must defeat him at the start of his attack, in the spirit of treading him down with the feet, so that he cannot rise again to the attack.\\

"Treading" does not simply mean treading with the feet. Tread with the body, tread with the spirit, and, of course, tread and cut with the long sword. You must achieve the spirit of not allowing the enemy to attack a second time. This is the spirit of forestalling in every sense. Once at the enemy, you should not aspire just to strike him, but to cling after the attack. You must study this deeply.\\
\section{To Know "Collapse"}

Everything can collapse. Houses, bodies, and enemies collapse when their rhythm becomes deranged.\\

In large-scale strategy, when the enemy starts to collapse, you must pursue him without letting the chance go. If you fail to take advantage of your enemies' collapse, they may recover.\\

In single combat, the enemy sometimes loses timing and collapses. If you let this opportunity pass, he may recover and not be so negligent thereafter. Fix your eye on the enemy's collapse, and chase him, attacking so that you do not let him recover. You must do this. The chasing attack is with a strong spirit. You must utterly cut the enemy down so that he does not recover his position. You must understand how to utterly cut down the enemy.\\
\section{To Become the Enemy}

"To become the enemy" means to think yourself in the enemy's position. In the world people tend to think of a robber trapped in a house as a fortified enemy. However, if we think of "becoming the enemy", we feel that the whole world is against us and that there is no escape. He who is shut inside is a pheasant. He who enters to arrest is a hawk. You must appreciate this.\\

In large-scale strategy, people are always under the impression that the enemy is strong, and so tend to become cautious. But if you have good soldiers, and if you understand the principles of strategy, and if you know how to beat the enemy, there is nothing to worry about.\\

In single combat also you must put yourself in the enemy's position. If you think, "Here is a a master of the Way, who knows the principles of strategy", then you will surely lose. You must consider this deeply.\\
\section{To Release Four Hands}

"To release four hands" is used when you and the enemy are contending with the same spirit, and the issue cannot be decided. Abandon this spirit and win through an alternative resource.\\

In large-scale strategy, when there is a "four hands" spirit, do not give up - it is man's existence. Immediately throw away this spirit and win with a technique the enemy does not expect.\\

In single combat also, when we think we have fallen into the "four hands" situation, we must defeat the enemy by changing our mind and applying a suitable technique according to his condition. You must be able to judge this.\\
\section{To Move the Shade}

"To move the shade" is used when you cannot see the enemy's spirit.\\

In large-scale strategy, when you cannot see the enemy's position, indicate that you are about to attack strongly, to discover his resources. It is easy then to defeat him with a different method once you see his resources.\\

In single combat, if the enemy takes up a rear or side attitude of the long sword so that you cannot see his intention, make a feint attack, and the enemy will show his long sword, thinking he sees your spirit. Benefiting from what you are shown, you can win with certainty. If you are negligent you will miss the timing. Research this well.\\
\section{To Hold Down a Shadow}

"Holding down a shadow" is use when you can see the enemy's attacking spirit.\\

In large-scale strategy, when the enemy embarks on an attack, if you make a show of strongly suppressing his technique, he will change his mind. Then, altering your spirit, defeat him by forestalling him with a Void spirit.\\

Or, in single combat, hold down the enemy's strong intention with a suitable timing, and defeat him by forestalling him with this timing. You must study this well.\\
\section{To Pass On}

Many things are said to be passed on. Sleepiness can be passed on, and yawning can be passed on. Time can be passed on also.\\

In large-scale strategy, when the enemy is agitated and shows an inclination to rush, do not mind in the least. Make a show of complete calmness, and the enemy will be taken by this and will become relaxed. When you see that this spirit has been passed on, you can bring about the enemy's defeat by attacking strongly with a Void spirit.\\

In single combat, you can win by relaxing your body and spirit and then, catching on to the moment the enemy relaxes, attack strongly and quickly, forestalling him.\\

What is know as "getting someone drunk" is similar to this. You can also infect the enemy with a bored, careless, or weak spirit. You must study this well.\\
\section{To Cause Loss of Balance}

Many things can cause a loss of balance. One cause is danger, another is hardship, and another is surprise. You must research this.\\

In large-scale strategy it is important to cause loss of balance. Attack without warning where the enemy is not expecting it, and while his spirit is undecided follow up your advantage and, having the lead, defeat him.\\

Or, in single combat, start by making a show of being slow, then suddenly attack strongly. Without allowing him space for breath to recover form the fluctuation of spirit, you must grasp the opportunity to win. Get the feel of this.
\section{To Frighten}

Fright often occurs, caused by the unexpected.\\

In large-scale strategy you can frighten the enemy not just by what you present to their eyes, but by shouting, making a small force seem large, or by threatening them from the flank without warning. These things all frighten. You can win by making best use of the enemy's frightened rhythm.\\

In single combat, also, you must use the advantage of taking the enemy unawares by frightening him with your body, long sword, or voice, to defeat him. You should research this well.\\
\section{To Soak In}

When you have come to grips and are striving together with the enemy, and you realize that you cannot advance, you "soak in" and become one with the enemy. You can win by applying a suitable technique while you are mutually entangled.\\

In battles involving large numbers as well as in fights with small numbers, you can often win decisively with the advantage of knowing how to "soak" into the enemy, whereas, were you to draw apart, you would lose the chance to win. Research this well.\\
\section{To Injure the Corners}

It is difficult to move strong things by pushing directly, so you should "injure the corners".\\

In large-scale strategy, it is beneficial to strike at the corners of the enemy's force. If the corners are overthrown, the spirit of the whole body will be overthrown. To defeat the enemy you must follow up the attack when the corners have fallen.\\

In single combat, it is easy to win once the enemy collapses. This happens when you injure the "corners" of his body, and thus weaken him. It is important to know how to do this, so you must research deeply.\\
\section{To Throw into Confusion}

This means making the enemy lose resolve.\\

In large-scale strategy we can use our troops to confuse the enemy on the field. Observing the enemy's spirit, we can make him think, "Here? There? Like that? Like this? Slow? Fast?". Victory is certain when the enemy is caught up in a rhythm which confuses his spirit.\\

In single combat, we can confuse the enemy by attacking with varied techniques when the chance arises. Feint a thrust or cut, or make the enemy think ou are going to close with him, and when he is confused you can easily win.\\

This is the essence of fighting, and you must research it deeply.\\
\section{The Three Shouts}

The three shouts are divided thus: before, during and after. Shout according to the situation. The voice is a thing of life. We shout against fires and so on, against the wind and the waves. The voice shows energy.\\

In large-scale strategy, at the start of battle we shout as loudly as possible. During the fight, the voice is low-pitched, shouting out as we attack. After the contest, we shout in the wake of our victory. These are the three shouts.\\

In single combat, we make as if to cut and shout "Ei!" at the same time to disturb the enemy, then in the wake of our shout we cut with the long sword. We shout after we have cut down the enemy - this is to announce victory. This is called "sen go no koe" (before and after voice). We do not shout simultaneously with flourishing the long sword. We shout during the fight to get into rhythm. Research this deeply.\\
\section{To Mingle}

In battles, when the armies are in confrontation, attack the enemy's strong points and, when you see that they are beaten back, quickly separate and attack yet another strong point on the periphery of his force. The spirit of this is like a winding mountain path.\\

This is an important fighting method for one man against many. Strike down the enemies in one quarter, or drive them back, then grasp the timing and attack further strong points to right and left, as if on a winding mountain path, weighing up the enemies' disposition. When you know the enemies' level attack strongly with no trace of retreating spirit.\\

What is meant by "mingling" is the spirit of advancing and becoming engaged with the enemy, and not withdrawing even one step. You must understand this.\\
\section{To Crush}

This means to crush the enemy regarding him as being weak.\\

In large-scale strategy, when we see that the enemy has few men, or if he has many men but his spirit is weak and disordered, we knock the hat over his eyes, crushing him utterly. If we crush lightly, he may recover. You must learn the spirit of crushing as if with a hand-grip.\\

In single combat, if the enemy is less skilful than ourself, if his rhythm is disorganised, or if he has fallen into evasive or retreating attitudes, we must crush him straightaway, with no concern for his presence and without allowing him space for breath. It is essential to crush him all at once. The primary thing is not to let him recover his position even a little. You must research this deeply.\\
\section{The Mountain-Sea Change}

The "mountain-sea" spirit means that it is bad to repeat the same thing several times when fighting the enemy. There may be no help but to do something twice, but do not try it a third time. If you once make an attack and fail, there is little chance of success if you use the same approach again. If you attempt a technique which you have previously tried unsuccessfully and fail yet again, then you must change your attacking method.\\

If the enemy thinks of the mountains, attack like the sea; and if he thinks of the sea, attack like the mountains. You must research this deeply.\\
\section{To Penetrate the Depths}

When we are fighting with the enemy, even when it can be seen that we can win on the surface with the benefit of the Way, if his spirit is not extinguished, he may be beaten superficially yet undefeated in spirit deep inside. With this principle of "penetrating the depths" we can destroy the enemy's spirit in its depths, demoralising him by quickly changing our spirit. This often occurs.\\

Penetrating the depths means penetrating with the long sword, penetrating with the body, and penetrating with the spirit. This cannot be understood in a generalisation.\\

Once we have crushed the enemy in the depths, there is no need to remain spirited. But otherwise we must remain spirited. If the enemy remains spirited it is difficult to crush him. You must train in penetrating the depths for large-scale strategy and also single combat.\\
\section{To Renew}

"To renew" applies when we are fighting with the enemy, and an entangled spirit arises where there is no possible resolution. We must abandon our efforts, think of the situation in a fresh spirit then win in the new rhythm. To renew, when we are deadlocked with the enemy, means that without changing our circumstance we change our spirit and win through a different technique.\\

It is necessary to consider how "to renew" also applies in large-scale strategy. Research this diligently.\\
\section{Rat's Head, Ox's Neck}

"Rat's head and ox's neck" means that, when we are fighting with the enemy and both he and we have become occupied with small points in an entangled spirit, we must always think of the Way of Strategy as being both a rat's head and an ox's neck. Whenever we have become preoccupied with small detail, we must suddenly change into a large spirit, interchanging large with small.\\

This is one of the essences of strategy. It is necessary that the warrior think in this spirit in everyday life. You must not depart from this spirit in large-scale strategy nor in single combat.\\
\section{The Commander Knows the Troops}

"The commander knows the troops" applies everywhere in fights in my Way of strategy.\\

Using the wisdom of strategy, think of the enemy as your own troops. When you think in this way you can move him at will and be able to chase him around. You become the general and the enemy becomes your troops. You must master this.\\
\section{To Let Go the Hilt}

There are various kinds of spirit involved in letting go the hilt.\\

There is the spirit of winning without a sword. There is also the spirit of holding the long sword but not winning. The various methods cannot be expressed in writing. You must train well.\\
\section{The Body of a Rock}

When you have mastered the Way of Strategy you can suddenly make your body like a rock, and ten thousand things cannot touch you. This is the body of a rock.\\

You will not be moved. Oral tradition.\\

What is recorded above is what has been constantly on my mind about Ichi school sword fencing, written down as it came to me. This is the first time I have written about my technique, and the order of things is a bit confused. It is difficult to express it clearly.\\

This book is a spiritual guide for the man who wishes to learn the Way.\\

My heart has been inclined to the Way of Strategy from my youth onwards. I have devoted myself to training my hand, tempering my body, and attaining the many spiritual attitudes of sword fencing. If we watch men of other schools discussing theory, and concentrating on techniques with the hands, even though they seem skilful to watch, they have not the slightest true spirit.\\

Of course, men who study in this way think they are training the body and spirit, but it is an obstacle to the true Way, and its bad influence remains for ever. Thus the true Way of Strategy is becoming decadent and dying out.\\

The true Way of sword fencing is the craft of defeating the enemy in a fight, and nothing other than this. If you attain and adhere to the wisdom of my strategy, you need never doubt that you will win. 
