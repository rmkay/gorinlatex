\chapter{THE WIND BOOK}

In strategy you must know the Ways of other schools, so I have written about various other traditions of strategys in this the Wind Book.\\

Without knowledge of the Ways of other schools, it is difficult to understand the essence of my Ichi school. Looking at other schools we find some that specialise in techniques of strength using extra-long swords. Some schools study the Way of the short sword, known as kodachi. Some schools teach dexterity in large numbers of sword techniques, teaching attitudes of the sword as the "surface" and the Way as the "interior".\\

That none of these are the true Way I show clearly in the interior of this book - all the vices and virtues and rights and wrongs. My Ichi school is different. Other schools make accomplishments their means of livelihood, growing flowers and decoratively colouring articles in order to sell them. This is definitely not the Way of Strategy.\\

Some of the world's strategists are concerned only with sword-fencing, and limit their training to flourishing the long sword and carriage of the body. But is dexterity alone sufficient to win? This is not the essence of the Way.\\

I have recorded the unsatisfactory point of other schools one by one in this book. You must study these matters deeply to appreciate the benefit of my Ni To Ichi school.\\
\section{Other Schools Using Extra-Long Swords}

Some other schools have a liking for extra-long swords. From the point of view of my strategy these must been seen as weak schools. This is because they do not appreciate the principle of cutting the enemy by any means. Their preference is for the extra-long sword and, relying on the virtue of its length, they think to defeat the enemy from a distance.\\

In this world it is said, "One inch gives the hand advantage", but these are the idle words of one who does not know strategy. It shows the inferior strategy of a weak spirit that men should be dependent on the length of their sword, fighting from a distance without the benefit of strategy.\\

I expect there is a case for the school in question liking extra-long swords as part of its doctrine, but if we compare this to real life it is unreasonable. Surely we need not necessarily be defeated if we are using a short sword, and have no long sword?\\

It is difficult for these people to cut the enemy when at close quarters because of the length of the long sword. The blade path is large so the long sword is an encumbrance, and they are at a disadvantage compared to the man armed with a short companion sword.\\

From olden times it has been said: "Great and small go together.". So do not unconditionally dislike extra-long swords. What I dislike is the inclination towards the long sword. If we consider large-scale strategy, we can think of large forces in terms of long swords, and small forces as short swords. Cannot few me give battle against many? There are many instances of few men overcoming many.\\

Your strategy is of no account if when called on to fight in a confined space your heart is inclined to the long sword, or if you are in a house armed only with your companion sword. Besides, some men have not the strength of others.\\

In my doctrine, I dislike preconceived, narrow spirit. You must study this well.\\
\section{The Strong Long Sword Spirit in Other Schools}

You should not speak of strong and weak long swords. If you just wield the long sword in a strong spirit your cutting will be coarse, and if you use the sword coarsely you will have difficulty in winning.\\

If you are concerned with the strength of your sword, you will try to cut unreasonably strongly, and will not be able to cut at all. It is also bad to try to cut strongly when testing the sword. Whenever you cross swords with an enemy you must not think of cutting him either strongly or weakly; just think of cutting and killing him. Be intent solely upon killing the enemy. Do not try to cut strongly and, of course, do not think of cutting weakly. You should only be concerned with killing the enemy.\\

If you rely on strength, when you hit the enemy's sword you will inevitably hit too hard. If you do this, your own sword will be carried along as a result. Thus the saying, "The strongest hand wins", has no meaning.\\

In large-scale strategy, if you have a strong army and are relying on strength to win, but the enemy also has a strong army, the battle will be fierce. This is the same for both sides.\\

Without the correct principle the fight cannot be won.\\

The spirit of my school is to win through the wisdom of strategy, paying no attention to trifles. Study this well.\\
\section{Use of the Shorter Long Sword in Other Schools}

Using a shorter long sword is not the true Way to win.\\

In ancient times, tachi and katana meant long and short swords. Men of superior strength in the world can wield even a long sword lightly, so there is no case for their liking the short sword. They also make use of the length of spears and halberds. Some men use a shorter long sword with the intention of jumping in and stabbing the enemy at the unguarded moment when he flourishes his sword. This inclination is bad.\\

To aim for the enemy's unguarded moment is completely defensive, and undesirable at close quarters with the enemy. Furthermore, you cannot use the method of jumping inside his defence with a short sword if there are many enemies. Some men think that if they go against many enemies with a shorter long sword they can unrestrictedly frisk around cutting in sweeps, but they have to parry cuts continuously, and eventually become entangled with the enemy. This is inconsistent with the true Way of Strategy.\\

The sure Way to win thus is to chase the enemy around in confusing manner, causing him to jump aside, with your body held strongly and straight. The same principle applies to large-scale strategy. The essence of strategy is to fall upon the enemy in large numbers and bring about his speedy downfall. By their study of strategy, people of the world get used to countering, evading and retreating as the normal thing. They become set in this habit, so can easily be paraded around by the enemy. The Way of Strategy is straight and true. You must chase the enemy around and make him obey your spirit.\\
\section{Other Schools with many Methods of using the Long Sword}

Placing a great deal of importance on the attitudes of the long sword is a mistaken way of thinking. What is known in the world as "attitude" applies when there is no enemy. The reason is that this has been a precedent since ancient times, and there should be no such thing as "This is the modern way to do it" in dueling. You must force the enemy into inconvenient situations.\\

Attitudes are for situations in which you are not to be moved. That is, for garrisoning castles, battle array, and so on, showing the spirit of not being moved even by a strong assault. In the Way of dueling, however, you must always be intent upon taking the lead and attacking. Attitude is the spirit of awaiting an attack. You must appreciate this.\\

In duels of strategy you must move the opponent's attitude. Attack where his spirit is lax, throw him into confusion, irritate and terrify him. Take advantage of the enemy's rhythm when he is unsettled and you can win.\\

I dislike the defensive spirit know as "attitude". Therefore, in my Way, there is something called "Attitude-No Attitude".\\

In large-scale strategy we deploy our troops for battle bearing in mind our strength, observing the enemy's numbers, and noting the details of the battle field. This is at the start of the battle.\\

The spirit of attacking first is completely different from the spirit of being attacked. Bearing an attack well, with a strong attitude, and parrying the enemy's attack well, is like making a wall of spears and halberds. When you attack the enemy, your spirit must go to the extent of pulling the stakes out of a wall and using them as spears and halberds. You must examine this well.\\
\section{Fixing the Eyes in Other Schools}

Some schools maintain that the eyes should be fixed on the enemy's long sword. Some schools fix the eyes on the hands. Some fix the eyes on the face, and some fix the eyes on the feet, and so on. If you fix the eyes on these places your spirit can become confused and your strategy thwarted.\\

I will explain this in detail. Footballers do not fix their eyes on the ball, but by good play on the field they can perform well. When you become accustomed to something, you are not limited to the use of your eyes. People such as master musicians have the music score in front of their nose, or flourish swords in several ways when they have mastered the Way, but this does not mean that they fix their eyes on these things specifically, or that they make pointless movements of the sword. It means that they can see naturally.\\

In the Way of Strategy, when you have fought many times you will easily be able to appraise the speed and position of the enemy's sword, and having mastery of the Way you will see the weight of his spirit. In strategy, fixing the eyes means gazing at the man's heart.\\

In large-scale strategy the area to watch is the enemy's strength. "Perception" and "sight" are the two methods of seeing. Perception consists of concentrating strongly on the enemy's spirit, observing the condition of the battlefield, fixing the gaze strongly, seeing the progress of the fight and the changes of advantages. This is the sure way to win.\\

In single combat you must not fix the eyes on the details. As I said before, if you fix your eyes on details and neglect important things, your spirit will become bewildered, and victory will escape you. Research this principle well and train diligently.\\
\section{Use of the Feet in Other Schools}

There are various methods of using the feet: floating foot, jumping foot, springing foot, treading foot, crow's foot, and such nimble walking methods. From the point of view of my strategy, these are all unsatisfactory.\\

I dislike floating foot because the feet always tend to float during the fight. The Way must be trod firmly.\\

Neither do I like jumping foot, because it encourages the habit of jumping, and a jumpy spirit. However much you jump, there is no real justification for it; so jumping is bad.\\

Springing foot causes a springing spirit which is indecisive.\\

Treading foot is a "waiting" method, and I especially dislike it.\\

Apart from these, there are various fast walking methods, such as crow's foot, and so on.\\

Sometimes, however, you may encounter the enemy on marshland, swampy ground, river valleys, stony ground, or narrow roads, in which situations you cannot jump or move the feet quickly.\\

In my strategy, the footwork does not change. I always walk as I usually do in the street. You must never lose control of your feet. According to the enemy's rhythm, move fast or slowly, adjusting you body not too much and not too little.\\

Carrying the feet is important also in large-scale strategy. This is because, if you attack quickly and thoughtlessly without knowing the enemy's spirit, your rhythm will become deranged and you will not be able to win. Or, if you advance too slowly, you will not be able to take advantage of the enemy's disorder, the opportunity to win will escape, and you will not be able to finish the fight quickly. You must win by seizing upon the enemy's disorder and derangement, and by not according him even a little hope of recovery. Practice this well.\\
\section{Speed in Other Schools}

Speed is not part of the true Way of Strategy. Speed implies that things seem fast or slow, according to whether or not they are in rhythm. Whatever the Way, the master of strategy does not appear fast.\\

Some people can walk as fast as a hundred or a hundred and twenty miles in a day, but this does not mean that they run continuously from morning till night. Unpracticed runners may seem to have been running all day, but their performance is poor.\\

In the Way of dance, accomplished performers can sing while dancing, but when beginners try this they slow down and their spirit becomes busy. The "old pine tree" melody beaten on a leather drum is tranquil, but when beginners try this they slow down and their spirit becomes busy. Very skilful people can manage a fast rhythm, but it is bad to beat hurriedly. If you try to beat too quickly you will get out of time. Of course, slowness is bad. Really skilful people never get out of time, and are always deliberate, and never appear busy. From this example, the principle can be seen.\\

What is known as speed is especially bad in the Way of Strategy. The reason for this is that depending on the place, marsh or swamp and so on, it may not be possible to move the body and legs together quickly. Still less will you be able to cut quickly if you have a long sword in this situation. If you try to cut quickly, as if using a fan or short sword, you will not actually cut even a little. You must appreciate this.\\

In large-scale strategy also, a fast busy spirit is undesirable. The spirit must be that of holding down a pillow, then you will not be even a little late.\\

When your opponent is hurrying recklessly, you must act contrarily and keep calm. You must not be influenced by the opponent. Train diligently to attain this spirit.\\
\section{"Interior" and "Surface" in Other Schools}

There is no "interior" nor "surface" in strategy.\\

The artistic accomplishments usually claim inner meaning and secret tradition, and "interior" and "gate", but in combat there is no such thing as fighting on the surface, or cutting with the interior. When I teach my Way, I first teach by training in techniques which are easy for the pupil to understand, a doctrine which is easy to understand. I gradually endeavour to explain the deep principle, points which it is hardly possible to comprehend, according to the pupil's progress. In any event, because the way to understanding is through experience, I do not speak of "interior" and "gate".\\

In this world, if you go into the mountains, and decide to go deeper and yet deeper, instead you will emerge at the gate. Whatever the Way, it has an interior, and it is sometimes a good thing to point out the gate. In strategy, we cannot say what is concealed and what is revealed.\\

Accordingly I dislike passing on my Way through written pledges and regulations. Perceiving the ability of my pupils, I teach the direct Way, remove the bad influence of other schools, and gradually introduce them to the true Way of the warrior.\\

The method of teaching my strategy is with a trustworthy spirit. You must train diligently.\\

I have tried to record an outline of the strategy of other schools in the above nine sections. I could now continue by giving a specific account of these schools one by one, from the "gate" to the "interior", but I have intentionally not named the schools or their main points. The reason for this is that different branches of schools give different interpretations of the doctrines. In as much as men's opinions differ, so there must be differing ideas on the same matter. Thus no one man's conception is valid for any school.\\

I have shown the general tendencies of other schools on nine points. If we look at them from an honest viewpoint, we see that people always tend to like long swords or short swords, and become concerned with strength in both large and small matters. You can see why I do not deal with the "gates" of other schools.\\

In my Ichi school of the long sword there is neither gate nor interior. There is no inner meaning in sword attitudes. You must simply keep your spirit true to realize the virtue of strategy. 
