\chapter{THE GROUND BOOK}

Strategy is the craft of the warrior. Commanders must enact the craft, and troopers should know this Way. There is no warrior in the world today who really understands the Way of Strategy.\\

There are various Ways. There is the Way of salvation by the law of Buddha, the Way of Confucius governing the Way of learning, the Way of healing as a doctor, as a poet teaching the Way of Waka, tea, archery, and many arts and skills. Each man practices as he feels inclined.\\

It is said the warrior's is the twofold Way of pen and sword, and he should have a taste for both Ways. Even if a man has no natural ability he can be a warrior by sticking assiduously to both divisions of the Way. Generally speaking, the Way of the warrior is resolute acceptance of death. Although not only warriors but priests, women, peasants and lowlier folk have been known to die readily in the cause of duty or out of shame, this is a different thing. The warrior is different in that studying the Way of Strategy is based on overcoming men. By victory gained in crossing swords with individuals, or enjoining battle with large numbers, we can attain power and fame for ourselves or our lord. This is the virtue of strategy.\\
\section{The Way of Strategy}

In China and Japan practitioners of the Way have been known as "masters of strategy". Warriors must learn this Way.\\

Recently there have been people getting on in the world as strategists, but they are usually just sword-fencers. The attendants of the Kashima Kantori shrines of the province Hitachi received instruction from the gods, and made schools based on this teaching, traveling from country to country instructing men. This is the recent meaning of strategy.\\

In olden times strategy was listed among the Ten Abilities and Seven Arts as a beneficial practice. It was certainly an art but as a beneficial practice it was not limited to sword-fencing. The true value of sword-fencing cannot be seen within the confines of sword-fencing technique.\\

If we look at the world we see arts for sale. Men use equipment to sell their own selves. As if with the nut and the flower, the nut has become less than th flower. In this kind of Way of Strategy, both those teaching and those learning the way are concerned with colouring and showing off their technique, trying to hasten the bloom of the flower. They speak of "This Dojo" and "That Dojo". They are looking for profit. Someone once said "Immature strategy is the cause of grief". That was a true saying.\\

There are four Ways in which men pass through life: as gentlemen, farmers, artisans and merchants.\\

The Way of the farmer. Using agricultural instruments, he sees springs through to autumns with an eye on the changes of season.\\

Second is the Way of the merchant. The wine maker obtains his ingredients and puts them to use to make his living. The Way of the merchant is always to live by taking profit. This is the Way of the merchant.\\

Thirdly the gentleman warrior, carrying the weaponry of his Way. The Way of the warrior is to master the virtue of his weapons. If a gentleman dislikes strategy he will not appreciate the benefit of weaponry, so must he not have a little taste for this?\\

Fourthly the Way of the artisan. The Way of the carpenter is to become proficient in the use of his tools, first to lay his plans with a true measure and then perform his work according to plan. Thus he passes through life. These are the four Ways of the gentleman, the farmer, the artisan and the merchant.\\
\section{Comparing the Way of the Carpenter to Strategy}

The comparison with carpentry is through the connection with houses. Houses of the nobility, houses of warriors, the Four houses, ruin of houses, thriving of houses, the style of the house, the tradition of the house, and the name of the house. The carpenter uses a master plan of the building, and the Way of Strategy is similar in that there is a plan of campaign. If you want to learn the craft of war, ponder over this book. The teacher is as a needle, the disciple is as thread. You must practice constantly.\\

Like the foreman carpenter, the commander must know natural rules, and the rules of the country, and the rules of houses. This is the Way of the foreman.\\

The foreman carpenter must know the architectural theory of towers and temples, and the plans of palaces, and must employ men to raise up houses. The Way of the foreman carpenter is the same as the Way of the commander of a warrior house.\\

In the construction of houses, choice of woods is made. Straight un-knotted timber of good appearance is used for the revealed pillars, straight timber with small defects is used for the inner pillars. Timbers of the finest appearance, even if a little weak, is used for the thresholds, lintels, doors, and sliding doors, and so on. Good strong timber, though it be gnarled and knotted, can always be used discreetly in construction. Timber which is weak or knotted throughout should be used as scaffolding, and later for firewood.\\

The foreman carpenter allots his men work according to their ability. Floor layers, makers of sliding doors, thresholds and lintels, ceilings and so on. Those of poor ability lay the floor joists, and those of lesser ability carve wedges and do such miscellaneous work. If the foreman knows and deploys his men well the finished work will be good.\\

The foreman should take into account the abilities and limitations of his men, circulating among them and asking nothing unreasonable. He should know their morale and spirit, and encourage them when necessary. This is the same as the principle of strategy.\\
\section{The Way of Strategy}

Like a trooper, the carpenter sharpens his own tools. He carries his equipment in his tool box, and works under the direction of his foreman. He makes columns and girders with an axe, shapes floorboards and shelves with a plane, cuts fine openwork and carvings accurately, giving as excellent a finish as his skill will allow. This is the craft of the carpenters. When the carpenter becomes skilled and understands measures he can become a foreman.\\

The carpenter's attainment is, having tools which will cut well, to make small shrines, writing shelves, tables, paper lanterns, chopping boards and pot-lids. These are the specialties of the carpenter. Things are similar for the trooper. You ought to think deeply about this.\\

The attainment of the carpenter is that his work is not warped, that the joints are not misaligned, and that the work is truly planed so that it meets well and is not merely finished in sections. This is essential.\\

If you want to learn this Way, deeply consider the things written in this book one at a time. You must do sufficient research.\\
\section{Outline of the Five Books of this Book of Strategy}

The Way is shown as five books concerning different aspects. These are Ground, Water, Fire, Wind (tradition), and Void (the illusionary nature of worldly things)\\

The body of the Way of Strategy from the viewpoint of my Ichi school is explained in the Ground book. It is difficult to realize the true Way just through sword-fencing. Know the smallest things and the biggest things, the shallowest things and the deepest things. As if it were a straight road mapped out on the ground, the first book is called the Ground book.\\

Second is the Water book. With water as the basis, the spirit becomes like water. Water adopts the shape of its receptacle, it is sometimes a trickle and sometimes a wild sea. Water has a clear blue colour. By the clarity, things of Ichi school are shown in this book.\\

If you master the principles of sword-fencing, when you freely beat one man, you beat any man in the world. The spirit of defeating a man is the same for ten million men. The strategist makes small things into big things, like building a great Buddha from a one foot model. I cannot write in detail how this is done. The principle of strategy is having one thing, to know ten thousand things. Things of Ichi school are written in this the Water book.\\

Third is the Fire book. This book is about fighting. The spirit of fire is fierce, whether the fire be small or big; and so it is with battles. The Way of battles is the same for man to man fights and for ten thousand a side battles. You must appreciate that spirit can become big or small. What is big is easy to perceive: what is small is difficult to perceive. In short, it is difficult for large numbers of men to change position, so their movements can be easily predicted. An individual can easily change his mind, so his movements are difficult to predict. You must appreciate this. The essence of this book is that you must train day and night in order to make quick decisions. In strategy it is necessary to treat training as part of normal life with your spirit unchanging. Thus combat in battle is described in the Fire book.\\

Fourthly the Wind book. This book is not concerned with my Ichi school but with other schools of strategy. By Wind I mean old traditions, present-day traditions, and family traditions of strategy. Thus I clearly explain the strategies of the world. This is tradition. It is difficult to know yourself if you do not know others. To all Ways there are side-tracks. If you study a Way daily, and your spirit diverges, you may think you are obeying a good Way but objectively it is not the true Way. If you are following the true way and diverge a little, this will later become a large divergence. You must realize this. Other strategies have come to be thought of as mere sword-fencing, and it is not unreasonable that this should be so. The benefit of my strategy, although it includes sword-fencing, lies in a separate principle. I have explained what is commonly meant by strategy in other schools in the Tradition (Wind) book.\\

Fifthly, the book of the Void. By void I mean that which has no beginning and no end. Attaining this principle means not attaining the principle. The Way of strategy is the Way of nature. When you appreciate the power of nature, knowing rhythm of any situation, you will be able to hit the enemy naturally and strike naturally. All this is the Way of the Void. I intend to show how to follow the true Way according to nature in the book of the Void.\\
\section{The Name Ichi Ryu Ni To (One school - two swords)}

Warriors, both commanders and troopers, carry two swords at their belt. In olden times these were called the long sword and the sword; nowadays they are known as the sword and the companion sword. Let it suffice to say that in our land, whatever the reason, a warrior carries two swords at his belt. It is the Way of the warrior.\\

"Nito Ichi Ryu" shows the advantages of using both swords.\\

The spear and the halberd are weapons which are carried out of doors.\\

Students of the Ichi school Way of Strategy should train from the start with the sword and the long sword in either hand. This is a truth: when you sacrifice your life, you must make fullest use of your weaponry. It is false not to do so, and to die with a weapon yet undrawn.\\

If you hold a sword with both hands, it is difficult to wield it freely to left and right, so my method is to carry the sword in one hand. This does not apply to large weapons such as the spear or halberd, but swords and companion swords can be carried in one hand. It is encumbering to hold a sword in both hands when you are on horseback, when running on uneven roads, on swampy ground, muddy rice fields, stony ground, or in a crowd of people. To hold the long sword in both hands is not the true Way, for if you carry a bow or spear or other arms in your left hand you have only one hand free for the long sword. However, when it is difficult to cut an enemy down with one hand, you must use both hands. It is not difficult to wield a sword in one hand; the Way to learn this is to train with two long swords, one in each hand. It will seem difficult at first, but everything is difficult at first. Bows are difficult to draw, halberds are difficult to wield; as you become accustomed to the bow so your pull will become stronger. When you become used to wielding the long sword, you will gain the power of the Way and wield the sword well.\\

As I will explain in the second book, the Water Book, there is no fast way of wielding the long sword. The long sword should be wielded broadly and the companion sword closely. This is the first thing to realize.\\

According to this Ichi school, you can win with a long weapon, and yet you can also win with a short weapon. In short, the Way of the Ichi school is the spirit of winning, whatever the weapon and whatever its size.\\

It is better to use two swords rather than one when you are fighting a crowd, and especially if you want to take a prisoner.\\

These things cannot be explained in detail. From one thing, know ten thousand things. When you attain the Way of Strategy there will not be one thing you cannot see. You must study hard.\\
\section{The Benefit of the Two Characters Reading "Strategy"}

Masters of the long sword are called strategists. As for the other military arts, those who master the bow are called archers, those who master the spear are called spearmen, those who master the gun are called marksmen, those who master the halberd are called halberdiers. But we do not call masters of the Way of the long sword "longswordsmen", nor do we speak of "companion swordsmen". Because bows, guns, spears and halberds are all warriors' equipment they are certainly part of strategy. To master the virtue of the long sword is to govern the world and oneself, thus the long sword is the basis of strategy. The principle is "strategy by means of the long sword". If he attains the virtue of the long sword, one man can beat ten men. Just as one man can beat ten, so a hundred men can beat a thousand, and a thousand can beat ten thousand. In my strategy, one man is the same as ten thousand, so this strategy is the complete warrior's craft.\\

The Way of the warrior does not include other Ways, such as Confucianism, Buddhism, certain traditions, artistic accomplishments and dancing. But even though these are not part of the Way, if you know the Way broadly you will see it in everything. Men must polish their particular Way.\\
\section{The Benefit of Weapons in Strategy}

There is a time and place for use of weapons.\\

The best use of the companion sword is in a confined space, or when you are engaged closely with an opponent. The long sword can be used effectively in all situations.\\

The halberd is inferior to the spear on the battlefield. With the spear you can take the initiative; the halberd is defensive. In the hands of one of two men of equal ability, the spear gives a little extra strength. Spear and halberd both have their uses, but neither is very beneficial in confined spaces. They cannot be used for taking a prisoner. They are essentially weapons for the field.\\

Anyway, if you learn "indoor" techniques, you will think narrowly and forget the true Way. Thus you will have difficulty in actual encounters.\\

The bow is tactically strong at the commencement of battle, especially battles on a moor, as it is possible to shoot quickly from among the spearmen. However, it is unsatisfactory in sieges, or when the enemy is more than forty yards away. For this reason there are nowadays few traditional schools of archery. There is little use nowadays for this kind of skill.\\

From inside fortifications, the gun has no equal among weapons. It is the supreme weapon on the field before the ranks clash, but once swords are crossed the gun becomes useless.\\

One of the virtues of the bow is that you can see the arrows in flight and correct your aim accordingly, whereas gunshot cannot be seen. You must appreciate the importance of this.\\

Just as a horse must have endurance and no defects, so it is with weapons. Horses should walk strongly, and swords and companion swords should cut strongly. Spears and halberds must stand up to heavy use, bows and guns must be sturdy. Weapons should be hardy rather than decorative.\\

You should not have a favourite weapon. To become over-familiar with one weapon is as much a fault as not knowing it sufficiently well. You should not copy others, but use weapons which you can handle properly. It is bad for commanders and troopers to have likes and dislikes. These are things you must learn thoroughly.\\
\section{Timing in Strategy}

There is timing in everything. Timing in strategy cannot be mastered without a great deal of practice.\\

Timing is important in dancing and pipe or string music, for they are in rhythm only if timing is good. Timing and rhythm are also involved in the military arts, shooting bows and guns, and riding horses. In all skills and abilities there is timing.\\

There is also timing in the Void.\\

There is timing in the whole life of the warrior, in his thriving and declining, in his harmony and discord. Similarly, there is timing in the Way of the merchant, in the rise and fall of capital. All things entail rising and falling timing. You must be able to discern this. In strategy there are various timing considerations. From the outset you must know the applicable timing and the inapplicable timing, and from among the large and small things and the fast and slow timings find the relevant timing, first seeing the distance timing and the background timing. This is the main thing in strategy. It is especially important to know the background timing, otherwise your strategy will become uncertain.\\

You win battles with the timing in the Void born of the timing of cunning by knowing the enemies' timing, and thus using a timing which the enemy does not expect.\\

All the five books are chiefly concerned with timing. You must train sufficiently to appreciate this.\\

If you practice day and night in the above Ichi school strategy, your spirit will naturally broaden. Thus is large scale strategy and the strategy of hand to hand combat propagated in the world. This is recorded for the first time in the five books of Ground, Water, Fire, Tradition (Wind), and Void. This is the way for men who want to learn my strategy:\\

    Do not think dishonestly.\\
    The Way is in training.\\
    Become acquainted with every art.\\
    Know the Ways of all professions.\\
    Distinguish between gain and loss in worldly matters.\\
    Develop intuitive judgement and understanding for everything.\\
    Perceive those things which cannot be seen.\\
    Pay attention even to trifles.\\
    Do nothing which is of no use. \\

It is important to start by setting these broad principles in your heart, and train in the Way of Strategy. If you do not look at things on a large scale it will be difficult for you to master strategy. If you learn and attain this strategy you will never lose even to twenty or thirty enemies. More than anything to start with you must set your heart on strategy and earnestly stick to the Way. You will come to be able to actually beat men in fights, and to be able to win with your eye. Also by training you will be able to freely control your own body, conquer men with your body, and with sufficient training you will be able to beat ten men with your spirit. When you have reached this point, will it not mean that you are invincible?\\

Moreover, in large scale strategy the superior man will manage many subordinates dextrously, bear himself correctly, govern the country and foster the people, thus preserving the ruler's discipline. If there is a Way involving the spirit of not being defeated, to help oneself and gain honour, it is the Way of strategy.
