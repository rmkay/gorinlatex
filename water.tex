\chapter{THE WATER BOOK}

The spirit of the Ni Ten Ichi school of strategy is based on water, and this Water Book explains methods of victory as the long-sword form of the Ichi school. Language does not extend to explaining the Way in detail, but it can be grasped intuitively. Study this book; read a word then ponder on it. If you interpret the meaning loosely you will mistake the Way.\\

The principles of strategy are written down here in terms of single combat, but you must think broadly so that you attain an understanding for ten-thousand-a-side battles.\\

Strategy is different from other things in that if you mistake the Way even a little you will become bewildered and fall into bad ways.\\

If you merely read this book you will not reach the Way of Strategy. Absorb the things written in this book. Do not just read, memorise or imitate, but so that you realize the principle from within your own heart study hard to absorb these things into your body.\\
\section{Spiritual Bearing in Strategy}

In strategy your spiritual bearing must not be any different from normal. Both in fighting and in everyday life you should be determined though calm. Meet the situation without tenseness yet not recklessly, your spirit settled yet unbiased. Even when your spirit is calm do not let your body relax, and when your body is relaxed do not let your spirit slacken. Do not let your spirit be influenced by your body, or your body be influenced by your spirit. Be neither insufficiently spirited nor over spirited. An elevated spirit is weak and a low spirit is weak. Do not let the enemy see your spirit.\\

Small people must be completely familiar with the spirit of large people, and large people must be familiar with the spirit of small people. Whatever your size, do not be misled by the reactions of your own body. With your spirit open and unconstricted, look at things from a high point of view. You must cultivate your wisdom and spirit. Polish your wisdom: learn public justice, distinguish between good and evil, study the Ways of different arts one by one. When you cannot be deceived by men you will have realized the wisdom of strategy.\\

The wisdom of strategy is different from other things. On the battlefield, even when you are hard-pressed, you should ceaselessly research the principles of strategy so that you can develop a steady spirit.\\
\section{Stance in Strategy}

Adopt a stance with the head erect, neither hanging down, nor looking up, nor twisted. Your forehead and the space between your eyes should not be wrinkled. Do not roll your eyes nor allow them to blink, but slightly narrow them. With your features composed, keep the line of your nose straight with a feeling of slightly flaring your nostrils. Hold the line of the rear of the neck straight: instill vigour into your hairline, and in the same way from the shoulders down through your entire body. Lower both shoulders and, without the buttocks jutting out, put strength into your legs from the knees to the tips of your toes. Brace your abdomen so that you do not bend at the hips. Wedge your companion sword in your belt against your abdomen, so that your belt is not slack - this is called "wedging in".\\

In all forms of strategy, it is necessary to maintain the combat stance in everyday life and to make your everyday stance your combat stance. You must research this well.\\
\section{The Gaze in Strategy}

The gaze should be large and broad. This is the twofold gaze "Perception and Sight". Perception is strong and sight week.\\

In strategy it is important to see distant things as if they were close and to take a distanced view of close things. It is important in strategy to know the enemy's sword and not to be distracted by insignificant movements of his sword. You must study this. The gaze is the same for single combat and for large-scale strategy.\\

It is necessary in strategy to be able to look to both sides without moving the eyeballs. You cannot master this ability quickly. Learn what is written here; use this gaze in everyday life and do not vary it whatever happens.\\
\section{Holding the Long Sword}

Grip the long sword with a rather floating feeling in your thumb and forefinger, with the middle finger neither tight nor slack, and with the last two fingers tight. It is bad to have play in your hands.\\

When you take up a sword, you must feel intent on cutting the enemy. As you cut an enemy you must not change your grip, and your hands must not "cower". When you dash the enemy's sword aside, or ward it off, or force it down, you must slightly change the feeling in your thumb and forefinger. Above all, you must be intent on cutting the enemy in the way you grip the sword.\\

The grip for combat and for sword-testing is the same. There is no such thing as a "man-cutting grip".\\

Generally, I dislike fixedness in both long swords and hands. Fixedness means a dead hand. Pliability is a living hand. You must bear this in mind.\\
\section{Footwork}

With the tips of your toes somewhat floating, tread firmly with your heels. Whether you move fast or slow, with large or small steps, your feet must always move as in normal walking. I dislike the three walking methods know as "jumping-foot", "floating-foot" and "fixed-steps".\\

So-called "Yin-Yang foot" is important in the Way. Yin-Yang foot means not moving only one foot. It means moving your feet left-right and right-left when cutting, withdrawing, or warding off a cut. You should not move on one foot preferentially.\\
\section{The Five Attitudes}

The five attitudes are: Upper, Middle, Lower, Right Side, and Left Side. These are the give. Although attitude has these five divisions, the one purpose of all of them is to cut the enemy. There are none but these five attitudes.\\

Whatever attitude you are in, do not be conscious of making the attitude; think only of cutting.\\

Your attitude should be large or small according to the situation. Upper, Lower and Middle attitudes are decisive. Left Side and Right Side attitudes are fluid. Left and Right attitudes should be used if there is an obstruction overhead or to one side. The decision to use Left or Right depends on the place.\\

The essence of the Way is this. To understand attitude you must thoroughly understand the middle attitude. The middle attitude is the heart of attitudes. If we look at strategy on a broad scale, the Middle attitude is the seat of the commander, with the other four attitudes following the commander. You must appreciate this.\\
\section{The Way of the Long Sword}

Knowing the Way of the long sword means we can wield with two fingers the sword we usually carry. If we know the path of the sword well, we can wield it easily.\\

If you try to wield the long sword quickly you will mistake the Way. To wield the long sword well you must wield it calmly. If you try to wield it quickly, like a folding fan or a short sword, you will err by using "short sword chopping". You cannot cut down a man with a long sword using this method.\\

When you have cut downwards with the longsword, lift it straight upwards; when you cut sideways, return the sword along a sideways path. Return the sword in a reasonable way, always stretching the elbows broadly. Wield the sword strongly. This is the Way of the longsword.\\

If you learn to use the five approaches of my strategy, you will be able to wield a sword well. You must train constantly.\\
\section{The Five Approaches}

    The first approach is the Middle attitude. Confront the enemy with the point of your sword against his face. When he attacks, dash his sword to the right and "ride" it. Or, when the enemy attacks, deflect the point of his sword by hitting downwards, keep your long sword where it is, and as the enemy renews his attack cut his arms from below. This is the first method. \\

The five approaches are this kind of thing. You must train repeatedly using a long sword in order to learn them. When you master my Way of the long sword, you will be able to control any attack the enemy makes. I assure you, there are no attitudes other than the five attitudes of the long sword of Ni To.\\

    In the second approach with the long sword, from the Upper attitude cut the enemy just as he attacks. If the enemy evades the cut, keep your sword where it is and, scooping up from below, cut him as he renews the attack. It is possible to repeat the cut from here. \\

In this method there are various changes in timing and spirit. You will be able to understand this by training in the Ichi school. You will always win with the five long sword methods. You must train repetitively.\\

    In the third approach, adopt the Lower attitude, anticipating scooping up. When the enemy attacks, hit his hands from below. As you do so he may try to hit your sword down. If this is the case, cut his upper arm(s) horizontally with a feeling of "crossing". This means that from the lower attitudes you hit the enemy at the instant that he attacks. \\

You will encounter this method often, both as a beginner and in later strategy. You must train holding a long sword.\\

    In this fourth approach, adopt the Left Side attitude. As the enemy attacks hit his hands from below. If as you hit his hands he attempts to dash down your sword, with the feeling of hitting his hands, parry the path of his long sword and cut across from above your shoulder. \\

This is the Way of the long sword. Through this method you win by parrying the line of the enemy's attack. You must research this.\\

    In the fifth approach, the sword is in the Right Side attitude. In accordance with the enemy's attack, cross your long sword from below at the side to the Upper attitude. Then cut straight from above. \\

This method is essential for knowing the Way of the long sword well. If you can use this method, you can freely wield a heavy long sword.\\

I cannot describe in detail how to use these five approaches. You must become well acquainted with my "in harmony with the long sword" Way, learn large-scale timing, understand the enemy's long sword, and become used to the five approaches from the outset. You will always win by using these five methods, with various timing considerations discerning the enemy's spirit. You must consider all this carefully.\\
\section{The "Attitude No-Attitude" Teaching}

"Attitude No-Attitude" means that there is no need for what are know as long sword attitudes.\\

Even so, attitudes exist as the five ways of holding the long sword. However you hold the sword it must be in such a way that it is easy to cut the enemy well, in accordance with the situation, the place, and your relation to the enemy. From the Upper attitude as your spirit lessens you can adopt the Middle attitude, and from the Middle attitude you can raise the sword a little in your technique and adopt the Upper attitude. From the lower attitude you can raise the sword and adopt the Middle attitudes as the occasion demands. According to the situation, if you turn your sword from either the Left Side or Right Side attitude towards the centre, the Middle or the Lower attitude results.\\

The principle of this is called "Existing Attitude - Nonexisting Attitude".\\

The primary thing when you take a sword in your hands is your intention to cut the enemy, whatever the means. Whenever you parry, hit, spring, strike or touch the enemy's cutting sword, you must cut the enemy in the same movement. It is essential to attain this. If you think only of hitting, springing, striking or touching the enemy, you will not be able actually to cut him. More than anything, you must be thinking of carrying your movement through to cutting him. You must thoroughly research this.\\

Attitude in strategy on a larger scale is called "Battle Array". Such attitudes are all for winning battles. Fixed formation is bad. Study this well.\\
\section{To Hit the Enemy "In One Timing"}

"In One Timing" means, when you have closed with the enemy, to hit him as quickly and directly as possible, without moving your body or settling your spirit, while you see that he is still undecided. The timing of hitting before the enemy decides to withdraw, break or hit, is this "In One Timing".\\

You must train to achieve this timing, to be able to hit in the timing of an instant.\\
\section{The "Abdomen Timing of Two"}

When you attack and the enemy quickly retreats, as you see him tense you must feint a cut. Then, as he relaxes, follow up and hit him. This is the "Abdomen Timing of Two".\\

It is very difficult to attain this by merely reading this book, but you will soon understand with a little instruction.\\
\section{No Design, No Conception}

In this method, when the enemy attacks and you also decide to attack, hit with your body, and hit with your spirit, and hit from the Void with your hands, accelerating strongly. This is the "No Design, No Conception" cut.\\

This is the most important method of hitting. It is often used. You must train hard to understand it.\\
\section{The Flowing Water Cut}

The "Flowing Water Cut" is used when you are struggling blade to blade with the enemy. When he breaks and quickly withdraws trying to spring with his long sword, expand your body and spirit and cut him as slowly as possible with your long sword, following your body like stagnant water. You can cut with certainty if you learn this. You must discern the enemy's grade.\\
\section{Continuous Cut}

When you attack and the enemy also attacks, and your swords spring together, in one action cut his head, hands and legs. When you cut several places with one sweep of the long sword, it is the "Continuous Cut". You must practice this cut; it is often used. With detailed practice you should be able to understand it.\\
\section{The Fire and Stones Cut}

The Fires and Stones Cut means that when the enemy's long sword and your long sword clash together you cut as strongly as possible without raising the sword even a little. This means cutting quickly with the hands, body and legs - all three cutting strongly. If you train well enough you will be able to strike strongly.\\
\section{The Red Leaves Cut}

The Red Leaves Cut [allusion to falling, dying leaves] means knocking down the enemy's long sword. The spirit should be getting control of his sword. When the enemy is in a long sword attitude in front of you and intent on cutting, hitting and parrying, you strongly hit the enemy's long sword with the Fire and Stones Cut, perhaps in the spirit of the "No Design, No Conception" Cut. If you then beat down the point of his sword with a sticky feeling, he will necessarily drop the sword. If you practice this cut it becomes easy to make the enemy drop his sword. You must train repetitively.\\
\section{The Body in Place of the Long Sword}

Also "the long sword in place of the body". Usually we move the body and the sword at the same time to cut the enemy. However, according to the enemy's cutting method, you can dash against him with your body first, and afterwards cut with the sword. If his body is immoveable, you can cut first with the long sword, but generally you hit first with the body and then cut with the long sword. You must research this well and practice hitting.\\
\section{Cut and Slash}

To cut and to slash are two different things. Cutting, whatever form of cutting it is, is decisive, with a resolute spirit. Slashing is nothing more than touching the enemy. Even if you slash strongly, and even if the enemy dies instantly, it is slashing. When you cut, your spirit is resolved. You must appreciate this. If you first slash the enemy's hands or legs, you must then cut strongly. Slashing is in spirit the same as touching. When you realize this, they become indistinguishable. Learn this well.\\
\section{Chinese Monkey's Body}

The Chinese Monkey's Body is the spirit of not stretching out your arms. The spirit is to get in quickly, without in the least extending your arms, before the enemy cuts. If you are intent upon not stretching out your arms you are effectively far away, the spirit is to go in with your whole body. When you come to within arm's reach it becomes easy to move your body in. You must research this well.\\
\section{Glue and Lacquer Emulsion Body}

The spirit of "Glue and Lacquer Emulsion Body" is to stick to the enemy and not separate from him. When you approach the enemy, stick firmly with your head, body and legs. People tend to advance their head and legs quickly, but their body lags behind. You should stick firmly so that there is not the slightest gap between the enemy's body and your body. You must consider this carefully.\\
\section{To Strive for Height}

By "to strive for height" is meant, when you close with the enemy, to strive with him for superior height without cringing. Stretch your legs, stretch your hips, and stretch your neck face to face with him. When you think you have won, and you are the higher, thrust in strongly. You must learn this.\\
\section{To Apply Stickiness}

When the enemy attacks and you also attack with the long sword, you should go in with a sticky feeling and fix your long sword against the enemy's as you receive his cut. The spirit of stickiness is not hitting very strongly, but hitting so that the long swords do not separate easily. It is best to approach as calmly as possible when hitting the enemy's long sword with stickiness. The difference between "Stickiness" and "Entanglement" is that stickiness is firm and entanglement is weak. You must appreciate this.\\
\section{The Body Strike}

The Body Strike means to approach the enemy through a gap in his guard. The spirit is to strike him with your body. Turn your face a little aside and strike the enemy's breast with your left shoulder thrust out. Approach with the spirit of bouncing the enemy away, striking as strongly as possible in time with yout breathing. If you achieve this method of closing with the enemy, you will be able to knock him ten or twenty feet away. It is possible to strike the enemy until he is dead. Train well.\\
\section{Three Ways to Parry His Attack}

There are three methods to parry a cut:\\

First, by dashing the enemy's long sword to your right, as if thrusting at his eyes, when he makes an attack.\\

Or, to parry by thrusting the enemy's long sword towards his right eye with the feeling of snipping his neck.\\

Or, when you have a short "long sword", without worrying about parrying the enemy's long sword, to close with him quickly, thrusting at his face with your left hand.\\

These are the three methods of parrying. You must bear in mind that you can always clench your left hand and thrust at the enemy's face with your fist. For this it is necessary to train well.\\
\section{To Stab at the Face}

To stab at the face means, when you are in confrontation with the enemy, that your spirit is intent of stabbing at his face, following the line of the blades with the point of your long sword. If you are intent on stabbing at his face, his face and body will become rideable. When the enemy becomes as if rideable, there are various opportunities for winning. You must concentrate on this. When fighting and the enemy's body becomes as if rideable, you can win quickly, so you ought not to forget to stab at the face. You must pursue the value of this technique through training.\\
\section{To Stab at the Heart}

To stab at the heart means, when fighting and there are obstructions above, or to the sides, and whenever it is difficult to cut, to thrust at the enemy. You must stab the enemy's breast without letting the point of your long sword waver, showing the enemy the ridge of the blade square-on, and with the spirit of deflecting his long sword. The spirit of this principle is often useful when we become tired or for some reason our long sword will not cut. You must understand the application of this method.\\
To Scold "Tut-TUT!"\\

"Scold" means that, when the enemy tries to counter-cut as you attack, you counter-cut again from below as if thrusting at him, trying to hold him down. With very quick timing you cut, scolding the enemy. Thrust up, "Tut!", and cut "TUT!" This timing is encountered time and time again in exchange of blows. The way to scold Tut-TUT is to time the cut simultaneously with raising your long sword as if to thrust the enemy. You must learn this through repetitive practice.\\
\section{The Smacking Parry}

By "smacking parry" is meant that when you clash swords with the enemy, you meet his attacking cut on your long sword with a tee-dum, tee-dum rhythm, smacking his sword and cutting him. The spirit of the smacking parry is not parrying, or smacking strongly, but smacking the enemy's long sword in accordance with his attacking cut, primarily intent on quickly cutting him. If you understand the timing of smacking, however hard your long swords clash together, your swordpoint will not be knocked back even a little. You must research sufficiently to realize this.\\
\section{There are Many Enemies}

"There are many enemies" applies when you are fighting one against many. Draw both sword and companion sword and assume a wide-stretched left and right attitude. The spirit is to chase the enemies around from side to side, even though they come from all four directions. Observe their attacking order, and go to meet first those who attack first. Sweep your eyes around broadly, carefully examining the attacking order, and cut left and right alternately with your swords. Waiting is bad. Always quickly re-assume your attitudes to both sides, cut the enemies down as they advance, crushing them in the direction from which they attack. Whatever you do, you must drive the enemy together, as if tying a line of fishes, and when they are seen to be piled up, cut them down strongly without giving them room to move.\\
\section{The Advantage when Coming to Blows}

You can know how to win through strategy with the long sword, but it cannot be clearly explained in writing. You must practice diligently in order to understand how to win.\\

Oral tradition: "The true Way of Strategy is revealed in the long sword."\\
\section{One Cut}

You can win with certainty with the spirit of "one cut". It is difficult to attain this if you do not learn strategy well. If you train well in this Way, strategy will come from your heart and you will be able to win at will. You must train diligently.\\
\section{Direct Communication}

The spirit of "Direct Communication" is how the true Way of the Ni To Ichi school is received and handed down.\\

Oral tradition: "Teach your body strategy."\\

Recorded in the above book is an outline of Ichi school sword-fighting.\\

To learn how to win with the long sword in strategy, first learn the five approaches and the five attitudes, and absorb the Way of the long sword naturally in your body. You must understand spirit and timing, handle the long sword naturally, and move body and legs in harmony with your spirit. Whether beating one man or two, you will then know values in strategy.\\

Study the contents of this book, taking one item at a time, and through fighting with enemies you will gradually come to know the principle of the Way.\\

Deliberately, with a patient spirit, absorb the virtue of all this, from time to time raising your hand in combat. Maintain this spirit whenever you cross swords with and enemy.\\

Step by step walk the thousand-mile road.\\

Study strategy over the years and achieve the spirit of the warrior. Today is victory over yourself of yesterday; tomorrow is your victory over lesser men. Next, in order to beat more skillful men, train according to this book, not allowing your heart to be swayed along a side-track. Even if you kill an enemy, if it is not based on what you have learned it is not the true Way.\\

If you attain this Way of victory, then you will be able to beat several tens of men. What remains is sword-fighting ability, which you can attain in battles and duels. 
